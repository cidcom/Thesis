%!TEX root = /Users/jakubkonka/Thesis/Thesis.tex
\chapter{FIX:ME Approximating...}
\label{cha:approximation}

\minitoc
\vspace{10mm}

This chapter explores whether an auction format represented by the DMP selling mechanism can be modeled as an asymmetric first-price sealed-bid auction (FPA) with common priors; that is, an auction in which bidders are characterized by different probability distributions sharing a common support. Ideally, if the DMP auction was shown to be a special case of some corresponding common priors auction, then the numerical solution methods presented in Hubbard and Parsch~\cite{HubbardPaarsch2011}, and extensively studied by the economic community, could be employed. Conversely, if the DMP auction format cannot be shown to constitute a special case of the common priors auction, then, provided the expected profits for the bidders in both auction formats differ only slightly, the common priors auction format could still effectively be used to approximate the solution to the DMP auction.

\section{Asymmetric First-Price Auction with Common Priors} % (fold)
\label{sec:asymmetric_first_price_auction_with_common_priors}

\subsection{Mathematical Description} % (fold)
\label{sub:mathematical_description}

% subsection mathematical_description (end)

\subsection{Numerical Solutions} % (fold)
\label{sub:numerical_solutions}

\subsubsection{Forward Shooting Method} % (fold)
\label{ssub:forward_shooting_method}

% subsubsection forward_shooting_method (end)

% subsection numerical_solutions (end)

% section asymmetric_first_price_auction_with_common_priors (end)

\section{Network Selection Mechanism Cast into Common Priors Setting} % (fold)
\label{sec:network_selection_mechanism_cast_into_common_priors_setting}

% section network_selection_mechanism_cast_into_common_priors_setting (end)

\section{Summary} % (fold)
\label{sec:summary_approximation}

% section summary_approximation (end)
