\documentclass[10pt,a4paper,notitlepage]{article}
\usepackage{amsmath,amsfonts,amsthm,amssymb,graphicx,url,fancyhdr,fancybox,fancyvrb}
\usepackage[intoc]{nomencl}
\usepackage{rotating}
\usepackage[margin=10pt,font=small,labelfont=bf]{caption}
\usepackage{subcaption}
\usepackage{wrapfig}
\usepackage{multirow}
\usepackage[nottoc]{tocbibind}
\usepackage{fullpage}
\usepackage{rotating}
\usepackage{xfrac}
\numberwithin{equation}{section}

%%%---Set line spread------------------------------------------------------------------
\linespread{1.5}
\pagestyle{plain}

% PDF SETUP------FILL IN HERE THE DOC TITLE AND AUTHOR---------------------------------
\usepackage[
      bookmarks,
      hypertexnames=false,
      breaklinks=true,
      pdftitle={List of Corrections},
      pdfauthor={Jakub Konka},
      plainpages=false]{hyperref}

\usepackage{parskip}

%%%---Start document-------------------------------------------------------------------
\begin{document}

\title{List of Corrections: PhD Viva}
\date{\today}
\author{Jakub Konka}

\maketitle

%% General
\setcounter{section}{0}
\renewcommand{\thesection}{G}
\renewcommand{\thesubsection}{G\arabic{subsection}}
\section{General Comments}

\setcounter{subsection}{0}
\subsection{Comment 1}
The presented work is not in the University Thesis Style---change. Thesis should conform to SU style.\\[-2ex]

\textbf{Response:}
The style was changed to match the University Thesis Style. More specifically:
\begin{enumerate}
	\item Title page now features University and Department;
	\item Lines are one-half spaced;
	\item Page with declaration of originality and ownership now features a heading;
\end{enumerate}

\subsection{Comment 2}
Section 2.2---you have slipped into first person English–––this should be changed throughout the entire thesis. Convert to third person English throughout. We/our.\\[-2ex]

\textbf{Response:}
All occurrences of ``we'' or ``our'' were removed and replaced with their passive voice counterpart.

\subsection{Comment 3}
Proofs should be in an appendix, not at the end of the chapter.\\[-2ex]

\textbf{Response:}
All proofs were moved to Appendix A. For convenience, each proposition proved in the Appendix is restated before the actual proof is conducted (numbering of propositions was preserved). Furthermore, in the thesis, after the first proposition is stated, the reader is directed to the Appendix for the proof.

\subsection{Comment 4}
Every equation should have a number.\\[-2ex]

\textbf{Response:}
All equations are now numbered.

\clearpage

%% Abstract
\setcounter{section}{0}
\renewcommand{\thesection}{A}
\renewcommand{\thesubsection}{A\arabic{subsection}}
\section{Abstract Comments}
\subsection{Comment 1}
Include the questions to be answered by the research.

\textbf{Response:}
FIX:ME

\subsection{Comment 2}
Is Digital Marketplace (DMP) place a well-known and well-defined term in communications research?

\textbf{Response:}
FIX:ME

\subsection{Comment 3}
What are the alternatives to the DMP?

\textbf{Response:}
FIX:ME

\subsection{Comment 4}
It is not clear what an equilibrium bidding strategy is from the text.

\textbf{Response:}
FIX:ME

\subsection{Comment 5}
Do market participants know what it means to operate in the most ``optimal'' way? Is this even possible?

\textbf{Response:}
FIX:ME

\subsection{Comment 6}
As written it is not clear whether the thesis offers any novel contributions.

\textbf{Response:}
FIX:ME

\clearpage

%% Chapter 1
\setcounter{section}{0}
\renewcommand{\thesection}{C\arabic{section}}
\renewcommand{\thesubsection}{C\arabic{section}.\arabic{subsection}}
\section{Chapter 1 Comments}
\subsection{Comment 1}
Objective 3---why simple? A model that facilitates... would be more appropriate.\\[-2ex]

\textbf{Response:}
Removed ``simple'' altogether.

\subsection{Comment 2}
What is a bearer service?\\[-2ex]

\textbf{Response:}
Bearer service, in telecommunications, is equivalent to data service; that is, a service that allows transmission of information between network interfaces. ``Bearer'' was removed from the text. Instead, some examples of services were added:
\begin{quote}
The exclusive one-to-one mapping between network operators and subscribers need no longer hold; when requesting a service (for example, making a phone call, or checking the email), the network selection mechanism will be responsible for selecting the network operator (access technology) that best matches the required quality requirements of the service.
\end{quote}

\subsection{Comment 3}
Are network operators interested in efficient network usage? Revenue generation may be stronger motivation?\\[-2ex]

\textbf{Response:}
Since both aspects are important (without efficient network usage, revenue may not be maximised), they are both included in the paragraph. The revised paragraph now reads:
\begin{quote}
From the network operators' perspective, the integration of wireless access technologies will allow for more efficient usage of network resources and improved revenue generation. In other words, it might be the most economic way of providing both universal coverage and broadband access [4].
\end{quote}

\subsection{Comment 4}
You do not make clear at the end of section 1.1 how the work conducted will be used (either in thesis or wider industry) nor how the contributions to knowledge will be gained.

\textbf{Response:}
FIX:ME

\subsection{Comment 5}
Will the economic analysis of the network selection in DMP be scalable beyond this construct?

\textbf{Response:}
FIX:ME

\subsection{Comment 6}
Is your work reported under Contribution 2 being used by anybody else? Chapter 1 needs more info on the DMP at this point in the thesis. Is it used today, is it just a theory (a concept that is not used in practice)?

\textbf{Response:}
FIX:ME

\subsection{Comment 7}
How is the work identified under your main contributions different to what is already in the public domain? You do not say and it is not clear from the presentation.

\textbf{Response:}
FIX:ME

\subsection{Comment 8}
I am not fully clear as to what other work is going on in this area based on the work presented in this thesis---I appreciate that some of this will come in subsequent chapters but a summary should be included in Chapter 1.

\textbf{Response:}
FIX:ME

\clearpage

%% Chapter 2
\section{Chapter 2 Comments}
\subsection{Comment 1}
Weird font sizes used in Fig.~2.3.\\[-2ex]

\textbf{Response:}
Unified font sizes.

\subsection{Comment 2}
Is the flow of network traffic important for the diagram in Fig.~2.1? Fig.~2.1: Ensure diagram is bidirectional.\\[-2ex]

\textbf{Response:}
Replaced ``lighting bolts'' with double pointed arrows in Fig.~2.1.

\subsection{Comment 3}
The first two paragraphs of section 2.1.3 are not needed.\\[-2ex]

\textbf{Response:}
Removed.

\subsection{Comment 4}
Section 3.2.1---auction theory is promising for what?\\[-2ex]

\textbf{Response:}
Auction theory is one of the most \emph{prominent} branches of economics.

\subsection{Comment 5}
Need a 4th level of headings.\\[-2ex]

\textbf{Response:}
Modified accordingly.

\subsection{Comment 6}
Will the Internet always be the source as identified in Fig. 2.1?\\[-2ex]

\textbf{Response:}
Yes. To make it clear, modified the reference of Fig.~2.1 so that it now references a similar figure in [4].

\subsection{Comment 7}
Why is the telecoms industry pushing to have an all IP based system?\\[-2ex]

\textbf{Response:}
Rephrased to explain the drive for an all-IP-based platform in more detail:
\begin{quote}
Concurrently, the industry is driving for an all-IP-based platform which enables integration of diverse access networks in a common scalable framework [9]. As a result, wide range of multimedia services can be extended to subscribers over \emph{heterogeneous wireless access networks}.
\end{quote}

\subsection{Comment 8}
You need to add a definition of utility in section 2.2.1 (or similar) and also that it can be difficult to quantify the utility.

\textbf{Response:}
FIX:ME

\subsection{Comment 9}
There is an over reliance on reference [11] in this chapter.

\textbf{Response:}
FIX:ME

\subsection{Comment 10}
So at the end of section 2.2.2 what point are you trying to make? Is this useful, has anybody used this in practice? What are the strengths and weaknesses?

\textbf{Response:}
FIX:ME

\subsection{Comment 11}
Why do you concentrate on games between subscribers and games between network operators, and not between operators and subscribers? P13: Why not a mixed game? Need a better justification. Are there advantages of approaching the problem as a mixed game?

\textbf{Response:}
FIX:ME

\subsection{Comment 12}
Why is the information you provide on games between subscribers and games between operators useful to your thesis? At the end of each of these unnumbered subsections (latex problem?) you do not say. Is it good or bad? What are the gaps in the research? How does it impact on your thesis? Some sections are merely descriptive---they need a point, e.g., Games Between Operators.

\textbf{Response:}
FIX:ME

\subsection{Comment 13}
Subsection 2.2.4---I agree that there are many different approaches on network selection but you do not give a very comprehensive literature review to support this. Furthermore, you do not make clear why you are dismissing [38 to 42] + other---existing literature review is not very extensive and is very directed towards one area: game theory.

\textbf{Response:}
FIX:ME

\subsection{Comment 14}
You do not make clear on P15 how your work is different to that reported in [8, 28], just that it is---this is missing.

\textbf{Response:}
FIX:ME

\subsection{Comment 15}
Need to demonstrate how DMP is more broadly applicable.

\textbf{Response:}
FIX:ME

\subsection{Comment 16}
Chapter 2 needs to make clear how this work extends the economic aspects of previously published work of other researchers.

\textbf{Response:}
FIX:ME

\clearpage

%% Chapter 3
\section{Chapter 3 Comments}
\subsection{Comment 1}
``report'' is a poor choice of variable.\\[-2ex]

\textbf{Response:}
Replaced with $\rho$.

\subsection{Comment 2}
P24: mention the paper via specific reference.\\[-2ex]

\textbf{Response:}
Rephrased and referenced the paper in question.

\subsection{Comment 3}
Section 3.2.2 contains too much design aspects---only reporting with relevance to the thesis is required. You can develop design choices elsewhere.\\[-2ex]

\textbf{Response:}
The design aspects of the reputation rating systems were added into the thesis after the reviewers of the IEEE Transactions on Vehicular Technology paper pointed out that this was missing. Furthermore, it is more descriptive than anything else, as the reputation rating system was developed by other researchers studying the DMP.

\subsection{Comment 4}
P8: More discussion on the definition of reputation is required. Does the reputation relate to the network operator or to each service offered by the network operator?\\[-2ex]

\textbf{Response:}
Added explanation in the paragraph describing the reputation rating system:
\begin{quote}
It is important to notice that the reputation rating relates to the network operator rather than to the type of service offered by the network operator, such as phone call, email or web browsing requests~[43].
\end{quote}

\subsection{Comment 5}
Is DMP a recognised term? Is the DMP used operationally or just a theoretical concept? Who developed the DMP concept?\\[-2ex]

\textbf{Response:}
No, it is a theoretical framework for management of communications services in a heterogeneous wireless environment. The concept was first proposed by Irvine~\emph{et al.}~in 2000. The answer to the questions were appropriately incorporated into the thesis. In particular, the introduction of DMP now reads:
\begin{quote}
The DMP is a theoretical market-based framework for trading wireless communications services. It was first proposed by Irvine~\emph{et al.}~in 2000~[8, 43], and it was developed with the heterogeneous wireless communications environment in mind, where the subscribers (of communications services) have the ability to select a network operator that reflects their preferences on a per-service basis.
\end{quote}

\subsection{Comment 6}
Why do concentrate on the simplest business model? Why not the most appropriate?\\[-2ex]

\textbf{Response:}
Changed ``simplest'' to ``basic''. Furthermore, added an explanation as to why the basic business model is the most appropriate:
\begin{quote}
While the original specification of the DMP advocates decoupling of service and network provision, the research reported in this thesis concentrates on the basic business model; i.e., service and network provision is handled by one entity, a network operator. The basic business model is seen as the most appropriate since any additional aspects of the decoupled case can later be incorporated into the mathematical model of the negotiation process developed in this thesis without affecting the results reported herein.
\end{quote}

\subsection{Comment 7}
More discussion on practicalities of substituting subscriber with service provider.\\[-2ex]

\textbf{Response:}
A discussion was added:
\begin{quote}
The applicability of the results reported in this thesis to the decoupled case deserves a more elaborate explanation. In the decoupled case, the subscribers do not enter into direct negotiation with the network providers. Instead, they are represented by the service providers who conduct the negotiation on their behalf; hence, acting as the buyers. While the results presented in this thesis are still applicable to this case, the overall business model of the DMP is significantly complicated by the fact that the service providers act as intermediaries between the subscribers and the network providers. For example, for the subcribers characterised by similar preferences, the service providers might enter into negotiation with the network providers only once rather than negotiate on behalf of each subscriber separately. Furthermore, the service providers might reach an agreement with the network providers that the winner of the negotiation will supply their services to a number of subscribers for a specified period of time, such as a day, a week, etc. The model developed in this thesis should incorporate these possibilities if it was extended to the decoupled case.
\end{quote}

\subsection{Comment 8}
It is not clear how to read Fig.~3.1 with respect to contracts.\\[-2ex]

\textbf{Response:}
In order to clarify the diagrams, ``Contractual relationship'' was renamed to ``Business relationship'' in Figures~3.1 and 3.2.

\subsection{Comment 9}
Contracts are missing from Fig.~3.2.\\[-2ex]

\textbf{Response:}
Made the figure more explicit by including subscriber and market provider, and all of the relationships between different entities.

\subsection{Comment 10}
References to common types of auction.

\textbf{Response:}
Added references to each paragraph describing common auction types.

\subsection{Comment 11}
End of Section 3.2.2---referencing in big blocks is totally inapproriate---you need to deal with these individually. This approach does not convey understanding. P24: decomposed large groups of references.

\textbf{Response:}
Removed the first block of references altogether since they are elaborated upon in the subsequent section. Decomposed the second block (regarding QoE) into a short paragraph summarising each paper in relation to QoE.
\begin{quote}
The literature on the concept of QoE abounds. For example, Kilkki discusses the conceptual differences between QoS and QoE, and how the concepts fit into the communications ecosystem~[49]. He further defines QoE to be a set of metrics, such as mean opinion score, that capture the experience of the users within a particular services. At the same time, he reserves QoS to consist of metrics capturing the quality of the service at the technical level between network and application; e.g., bit rates, delay properties, and packet loss rates. Brooks and Hestnes, on the other hand, discuss different way of measuring QoE both subjectively and objectively~[50]. Furthermore, Fiedler~\emph{et al.}~propose a quantitative formula for relating QoS with QoE~[51], and similarly, Shaikh~\emph{et al.}~provide insight into the correlations of QoS and QoE; hence, capture the relationships between the two metrics~[52].
\end{quote}

\subsection{Comment 12}
There are no references in subsection 3.2.1 and an overreliance on [8] elsewhere.

\textbf{Response:}
Section 3.2.1 features now more references (a reference for each auction type; see C3.10). Changed some of the [8] to [44] (LeBodic's thesis). These are the main sources of information about fundamental assumptions of the Digital Marketplace.

\subsection{Comment 13}
You do not provide any text on markets where a physical commodity is traded (e.g., electricity, capacity markets)---this is missing from the thesis as a standalone section to bring out the aspect of physical trade that are relevant or otherwise to your work. Need to mention some other markets, e.g., oil, gas, electricity, etc.

\textbf{Response:}
FIX:ME

\subsection{Comment 14}
You need to be clearer on why you have just provided the bidding strategies rather than developing them---so as far as I know from what you have presented in the thesis, the bidding strategies in [8] and [36] could just be random numbers!

\textbf{Response:}
The thesis now features an explanation:
\begin{quote}
It is important to realise that the authors do not verify whether the proposed bidding strategy constitutes an equilibrium of the network selection mechanism; rather, it is provided as is. From the economic perspective, this is major flaw in the analysis as otherwise it is unclear whether the mechanism induces rational behaviour in participants; for example, whether it maximises the network operators' expected utilities.
\end{quote}

\subsection{Comment 15}
Again taking your work from an economic point of view is one possible approach---why is it the most important?

\textbf{Response:}
A detailed explanation of why economic analysis of the DMP network selection mechanism is now included in the thesis:
\begin{quote}
It is important to analyse the network selection mechanism from the economic point of view (using game-theory or otherwise) in order to verify whether the mechanism performs as expected in economic terms. For example, thanks to the economic analysis presented in this thesis, it is verified that the DMP network selection mechanism maximises the expected utilities of the network operators, and from the subscriber's perspective, the network operator who matches the subscriber's preferences in terms of requested price and quality of service wins the auction. In particular, this thesis characterises expected bidding behaviour that constitutes an equilibrium of the auction upon which the network selection mechanism is based.
\end{quote}

\clearpage

%% Chapter 4
\section{Chapter 4 Comments}
\subsection{Comment 1}
Figs. 4.1 and 4.2---add colour plot.\\[-2ex]

\textbf{Response:}
Revised accordingly.

\subsection{Comment 2}
In Figs. 4.1 and 4.2 is $i$ and $j$ a proxy for 1 and 2? $n=2$ in this section.\\[-2ex]

\textbf{Response:}
Yes it is. It was revised accordingly; $i$ was renamed to 1 and $j$ to 2 both in the text and in the figures. The changes make it compatible with Section 5.3, Chapter 5.

\subsection{Comment 3}
Equation (4.7): $n$ was previously the number of network operators---is this now varying with $i$?\\[-2ex]

\textbf{Response:}
To clarify the exposition, the variables were renamed as follows: $m_i$ is now $\zeta_i$, and $n_i$ is now $\eta_i$. See Equation (4.26) in the revised manuscript.

\subsection{Comment 4}
P40: Need to explain $b'$.\\[-2ex]

\textbf{Response:}
The apostrophe was dropped to simplify the notation.

\subsection{Comment 5}
Eq.~(4.15): Assuming $b(1) = 1$, need to explain the implications of this. P33: does making the assumption $b(1)=1$ have a physical meaning?\\[-2ex]

\textbf{Response:}
Added explanation in the text:
\begin{quote}
Since $c_i\in [0,1]$ for all $i\in N$, it follows $b(1) = 1$. To see this, suppose network operator 1 is characterised by cost $c_1 = 1$. Then, they would never submit a bid higher than their cost $c_1 = 1$ since they would never win. That is, the competing network operator, network operator 2 say, regardless of their cost, could just bid $b(c_2) = c_1 = 1$ and win the auction. Furthermore, network operator 1 would never submit a bid lower than their cost $c_1 = 1$ since they would make a loss if they were to win the auction. Therefore, it must be that $b(1) = 1$.
\end{quote}

\subsection{Comment 6}
First equations on P29---will all the operations use the same weights?\\[-2ex]

\textbf{Response:}
Clarified in the paragraph before the equations:
\begin{quote}
It is, moreover, assumed that the price and reputation weights $(w_{price}, w_{penalty})$ are announced by the subscriber to all network operators before the auction. They are specific to the subscriber and the service they requested. In other words, it is envisaged that the same subscriber might use different weights for subsequently requested services during their participation in the DMP.
\end{quote}

\subsection{Comment 7}
What happens if a network operator bids below their cost? Need a discussion on practicalities of real bids (incl. loss leader pricing strategy). Does allowing negative bids have a physical interpretation?\\[-2ex]

\textbf{Response:}
The point was addressed in the thesis. A paragraph explaining the problem and assumptions of game theory was added to the end of section 4.1:
\begin{quote}
It is further assumed that the network operators will bid at least their cost (unless explicitly stated otherwise). The problem of bidding below cost or negatively deserves a more elaborate explanation. There are two fundamental assumptions governing game theory~[22]: 1) economic agents are rational decision-makers; that is, they make decisions consistently in pursuit of their own objectives; and 2) their objective is to maximise the expected value of their own utility. In the light of those assumptions, network operators are implied to bid at least their cost as they would always strive to maximise their expected utility. However, the real behaviour of the network operators might be different in the sense that they might, in the view of game theory, behave irrationally by bidding below their cost to secure the contract with the subscriber. In fact, if the temporal aspect of the DMP is considered, network operators will interact by engaging in the DMP auction more than once. Then, it might prove beneficial for them to bid below their cost trading positive utility for securing the win---this is a well-known pricing strategy in economics called ``loss leader'' pricing strategy~[61, 62]. The idea behind the strategy is to sell a good at a price below its market cost to increase the store traffic and encourage sales of other, possibly more profitable goods. Applied back to the DMP, a network operator could, in principle, willingly incur cost by bidding below their cost to encourage more subscribers to use their services, or to improve their reputation rating by serving more subscribers. While the fact that situations like this can occur in reality is appreciated, this thesis follows the fundamental assumptions of game theory; that is, in the rest of this thesis (unless explicitly stated otherwise), it is assumed that all economic agents involved in the DMP are rational decision-makers and strive to maximise their expected utility. Otherwise, the mathematical treatment of the problem would prove impossible~[63].
\end{quote}

\subsection{Comment 8}
You need to make clear that $b$ contains a symmetric bid/offer pair.

\textbf{Response:}
The notation is now properly explained:
\begin{quote}
where $b = (b_i,b_{-i})$ represents the monetary bid (or offered price) vector, $c = (c_i, c_{-i})$ the type vector, and $r = (r_i, r_{-i})$ the reputation rating vector. In this notation, $b_{-i}$ is a shorthand notation for a vector containing all elements with the $b_i$ excluded; that is, $b_{-i} = (b_1, \ldots, b_{i-1}, b_{i+1}, \ldots, b_n)$.
\end{quote}

\subsection{Comment 9}
Need a definition of a bid/offer. Relate the maths back to the application.

\textbf{Response:}
The definition of the bid/offer was already provided in Section~3.2.2, Chapter~3. However, to relate the mathematics back to the application, the definition is restated after the bid/offer is formally introduced in Section~4.1, Chapter~4:
\begin{quote}
Furthermore, as already stated in Section~3.2.2, Chapter~3, the monetary bid is equivalent to the price of supporting the service by the network operator. The precise definition of the price is left open-ended; one possibility, for example, would be to charge the buyer per unit of bandwidth.
\end{quote}

\subsection{Comment 10}
You should provide a better description of FPA before embarking on the mathematics. P28 need a quick primer of first-price sealed-bid auction and Bayesian games. Need to mention why it is a Bayesian game of this type.

\textbf{Response:}
The beginning of Section~4.1 now features a more detailed description of FPA and how it relates to Bayesian games:
\begin{quote}
The game-theoretic description of the network selection mechanism employed in the DMP is as follows. The model constitutes a version of procurement first-price sealed-bid auction (henceforth, referred to as FPA). To recap, in FPA, bidders submit their bids simultaneously~[23]. Furthermore, each bidder knows their own cost of selling the good to the buyer but does not know any other bidder's type; i.e., the costs are private knowledge. The bidder who submitted the highest bid, wins the auction and sells the good to the buyer for the price equivalent to their bid. Since the costs are private knowledge, the bidders are uncertain about another bidders' utility functions. Thus, FPA and the network selection mechanism represent Bayesian games of incomplete information (see Section~B.4, Appendix~B for the formal definition of a Bayesian game of incomplete information).
\end{quote}

\subsection{Comment 11}
Summary in Section 4.4 undersells what has been done. Need to relate to the wider, more general problem in the application domain.

\textbf{Response:}
The summary was rewritten and now includes more discussion about the applicability of the results to real life:
\begin{quote}
In this chapter, game-theoretical model for the DMP network selection mechanism was formally defined. Several simplifying assumptions were made in order to keep the analysis mathematically tractable. For example, network operators and the buyer are risk neutral, and the buyer does not have any budget constraints. Despite the fact that those assumptions are not entirely representative of the reality, following in the footsteps of von Neumann and Morgenstern~[65], the mathematical theory should be rigorous and developed gradually. Therefore, the simplifying assumptions made in this chapter serve as a starting point for the rigorous, gradual development of the theory of operation of the DMP network selection mechanism before it can embark on capturing the reality to a high degree.

This chapter further demonstrated that for the price weight of $w=1$, and equal reputation ratings for all network operators, $r_i=r_j$ for all $i\neq j$, the DMP auction reduces to the standard, symmetric FPA (Proposition~4.2 and Corollary~4.3). In this case, the abundance of theoretical results and economic insight from the auction literature applies found, for example, in Krishna~[47]. For the price weight of $w=0$, however, it was shown that the network operators would engage in abnormally high bidding (Proposition~4.1). Hence, charging the buyer the maximum they are prepared to pay for the service. While this result sounds like a potential design flaw in the DMP network selection mechanism, in reality, the buyers will necessarily be budget constrained and therefore, abnormally high bidding of the network operators will translate into charging the buyers a premium price for the service that is within the limits of their respective budgets.

Finally, the chapter concluded with the specification of an analytical solution to the restricted case of two network operators $n=2$ (Proposition~4.4). The solution is suboptimal in the sense that the derived equilibrium bidding strategies permit the network operators to bid negatively. In the view of game theory, this would imply that the network operators are not rational decision-makers. However, it was also shown that negative bidding does not lead to negative profit for either network operator (Proposition~4.5). Concurrently, it was proved that the network operators would not find it beneficial not to participate in the auction if they were to bid according to the strategies summarised in Proposition~4.4 (Proposition~4.6). It should further be noted that the real behaviour of the network operators might be dictated by the need to secure the contract with the subscriber first and foremost, and hence, lead to negative bidding; a strategy akin to the ``loss leader'' pricing strategy. However, since the ultimate aim of this thesis is to gradually develop rigorous theory of operation of the DMP auction, it is assumed throughout this thesis that network operators will bid at least their cost.
\end{quote}

\clearpage

%% Chapter 5
\section{Chapter 5 Comments}
\subsection{Comment 1}
Remove obsolete figures from Chapter 5.\\[-2ex]

\textbf{Response:}
The following figures were removed: 5.8, 5.10, 5.12, 5.14, 5.17 and 5.19. Appropriate changes in the sections referencing the figures were made.

\subsection{Comment 2}
Remove Section 5.4.4 (Chebyshev) entirely.\\[-2ex]

\textbf{Response:}
Section was removed and appropriate changes in the sections referencing the section were made.

\subsection{Comment 3}
Fig. 5.16---what is customised finite difference? Change to EFSM.\\[-2ex]

\textbf{Response:}
The caption of the figure was changed to EFSM (Fig. 5.12 in the revised manuscript). Furthermore, any mention of ``customised finite-differences'' method was removed from the text.

\subsection{Comment 4}
P66: ``in particular'' should be ``for example''.\\[-2ex]

\textbf{Response:}
Revised accordingly.

\subsection{Comment 5}
P65: mode detail on what ``slightly modified'' entails---just say ``using existing methods''.\\[-2ex]

\textbf{Response:}
Revised accordingly.

\subsection{Comment 6}
P56: should read network operator 1.\\[-2ex]

\textbf{Response:}
Revised accordingly.

\subsection{Comment 7}
P61: why do you make this statement regarding ``abuse of notation''?\\[-2ex]

\textbf{Response:}
Originally, it was introduced to justify the weird notation of the expected prices. However, since the notation is legible, the statement ``abuse of notation'' was removed.

\subsection{Comment 8}
P61: How do you know 10,000 is enough? Was this number empirically chosen? Need to justify. Why has 10,000 been chosen? Is it large?\\[-2ex]

\textbf{Response:}
A sentence explaining why 10,000 observations was added in the paragraph:
\begin{quote}
Furthermore, it was empirically established that averaging over more than 10,000 observations does not drastically improve the results; that is, the already narrow 95\% confidence intervals do not get narrower as the number of observations increases beyond 10,000. In other words, 10,000 is large enough a sample size, and therefore, an average of 10,000 observations of the price for each selected price weight should provide a reasonable approximation of the expected price for that price weight.
\end{quote}

\subsection{Comment 9}
Superscript notation in Fig.~5.4 not clear. Put superscripts in brackets so that it does not look like squaring.\\[-2ex]

\textbf{Response:}
Superscripts are now in brackets in the text and in Fig.~5.4.

\subsection{Comment 10}
Implications of adapting price weights should be discussed.\\[-2ex]

\textbf{Response:}
A few sentences discussing the implications were added to the appropriate paragraph:
\begin{quote}
In other words, for any expected price, as the difference $(r_2-r_1)$ between the reputation ratings of the network operators increases, the price weight has to increase (or remain constant) in order to keep the expected price fixed. This observation carries very serious implications on the operation of the DMP, as the subscriber is effectively given the ability to influence the expected prices by an appropriate choice of the price weight. To illustrate, suppose there are 2 network operators characterised by reputation ratings $(r_1, r_2)$. Suppose further that the subscriber paid the price of $p_1$ at some point in the past for some type of service, and they request the same service again. Therefore, in order to pay the expected price of at most $p_1$, the subscriber solves
\begin{equation}
p_1 \le E[p_1](w, r_1, r_2)
\end{equation}
for the price weight $w$. In this way, the subscriber is guaranteed the expected price of at most $p_1$.
\end{quote}

\subsection{Comment 11}
P68, P71, P82---how do these algorithms terminate?\\[-2ex]

\textbf{Response:}
Each algorithmic listing features ``input'' and ``output'' statements at the very top of the listing.

\subsection{Comment 12}
You have avoided using the $k$-tuple language elsewhere in the thesis, so why start on P69?\\[-2ex]

\textbf{Response:}
All occurrences of $k$-tuples were changed to vectors.

\subsection{Comment 13}
Weird ends in Figs.~5.12 and 5.13---any explanation?\\[-2ex]

\textbf{Response:}
The explanation is provided in the text:
\begin{quote}
In the FSM case, the approximation diverges in the very near proximity of $\bar{\hat{b}}$, and therefore, the approximation satisfies the sufficiency only until $b$ reaches a close neighbourhood of $\bar{\hat{b}}$. This is due to the fact that the system of ODEs does not satisfy Lipschitz condition as $b$ approaches $\bar{\hat{b}}$.
\end{quote}

\subsection{Comment 14}
Brackets on equation after (5.33) is strange. P81: need clearer notation.\\[-2ex]

\textbf{Response:}
The set-builder notation was removed altogether. Instead, the conditions were unwrapped into separate equations, and are reference in the algorithms and in the text.

\subsection{Comment 15}
What value are Kaplan and Zamir putting on the non-standard equilibria?\\[-2ex]

\textbf{Response:}
Added an explanation of the importance of the non-standard equilibria:
\begin{quote}
Kaplan and Zamir, who characterise equilibria like the one specified in Proposition~4.4 as non-standard, further argue that such equilibria are important and should not be neglected since they may result in different winning offered prices and different network operators winning the auction. In other words, relaxing the assumption about network operators submitting at least their cost might help in understanding the ``deviations'' from the predicted equilibrium bidding behaviour prescribed in Proposition~5.5 should these occur in reality. Here, those ``deviations'' may be captured by non-standard equilibria.
\end{quote}

\subsection{Comment 16}
P84: is this brevity?\\[-2ex]

\textbf{Response:}
Rephrased to say:
\begin{quote}
Similarly to the algorithms presented in Section~5.4, the EFSM method was tested for correct implementation using the same procedure as presented in Section~5.4. However, since the verification results match exactly those presented in Figure~5.6, their discussion is omitted from this section.
\end{quote}

\subsection{Comment 17}
What is the algorithm on P58 based on? Is it yours or taken from elsewhere? P58: Make clear whose procedure this is. Put into algorithmic style.[-2ex]

\textbf{Response:}
Added appropriate listing (Listing 5.1). Furthermore, updated the description of the steps:
\begin{quote}
Listing~5.1 depicts the pseudo-code of the proposed method. The steps of the algorithm can be summarised as follows:
\begin{enumerate}
\item For a particular price weight $w$ and reputation ratings $r_1$ and $r_2$, calculate the costs-hat supports for both network operators; that is, the endpoints of the interval $[\underline{\hat{c}}_1, \bar{\hat{c}}_1]$ for network operator 1, and $[\underline{\hat{c}}_2, \bar{\hat{c}}_2]$ for network operator 2 (lines 1--4).
\item If $\bar{\hat{c}}_1 \le 2\underline{\hat{c}}_2 - \bar{\hat{c}}_2$, then the equilibrium is trivial. Network operator 1 bids the lower endpoint of the cost-hat support of network operator 2; that is, network operator 1 bids $\underline{\hat{c}}_2$ for all $\hat{c}\in [\underline{\hat{c}}_1, \bar{\hat{c}}_1]$. Network operator 2, on the other hand, bids their cost-hat; that is, network operator 2 bids $\hat{c}$ for all $\hat{c}\in [\underline{\hat{c}}_2, \bar{\hat{c}}_2]$ (lines 6--9).
\item If $\bar{\hat{c}}_1 > 2\underline{\hat{c}}_2 - \bar{\hat{c}}_2$, then the equilibrium is nontrivial. Hence,
\begin{enumerate}
\item Calculate the common bids-hat support $[\underline{\hat{b}}, \bar{\hat{b}}]$ using Equation~(5.37) (lines 11--12).
\item For all $\hat{b}\in [\underline{\hat{b}}, \bar{\hat{b}}]$, calculate the corresponding costs-hat for both network operators using Equations~(5.33)~and~(5.34). Since by assumption $r_1 < r_2$, it follows that $\bar{\hat{c}}_1 \le \bar{\hat{c}}_2$, and hence, $\bar{\hat{b}}\le \bar{\hat{c}}_2$. Thus, network operator 2 bids their cost-hat, $\hat{c}_2(\hat{b}) = \hat{c}_2$ for all $\hat{b}\in [\bar{\hat{b}}, \bar{\hat{c}}_2]$ (lines 13--16).
\end{enumerate}
\end{enumerate}
\end{quote}

\subsection{Comment 18}
Can you explain the simplification made before equation (5.27) and its impact?\\[-2ex]

\textbf{Response:}
A sentence explaining the impact of the simplifaction was added:
\begin{quote}
The ultimate aim of the numerical analysis is to obtain a numerical approximation to the equilibrium bidding strategies for all bidding scenarios that involve feasible bidders (cf.~Definition~5.2). However, as shown below, the assumption~(5.46) restricts the choice of the price weight and the reputation ratings to a subset of all possible bidding scenarios involving feasible bidders.
\end{quote}

\subsection{Comment 19}
Section 5.4.1 you need to make clear what FSM is all about then move to the specifics of the problem of interest. Section 5.4.2---again you need to make clear what PPM is all about then move to the specifics of the problem of interest.\\[-2ex]

\textbf{Response:}
Removed the first paragraph from the sections which was confusing and delayed the description of the methods under consideration. Moved the paragraph to the end of the section preceding FSM and PPM sections.

\subsection{Comment 20}
Section 5.4---you do not make clear whether these types of equations (5.13) to (5.14) arise elsewhere in other domains. P63: do the ODEs appear anywhere else with the same structure. Perhaps a quick look would be useful.

\textbf{Response:}
To the best of my knowledge, the system of ODEs is unique to auction theory domain. An explanation was added to the thesis to make it clear:
\begin{quote}
It is worth noting that, to the best of the author's knowledge, the system of ODEs in Equation~(5.27) with boundary conditions~(5.28)~and~(5.29) is unique to the domain of auction theory. This can be attributed to the fact that the derivation of the system involves the inverses of the equilibrium bidding strategy functions (as opposed to the equilibrium bidding strategy functions themselves), and unknown \emph{a priori} lower bound on bids, $\underline{\hat{b}}$.
\end{quote}

\subsection{Comment 21}
Why have you chosen to mention finite differences on P64? This is just one possibility + you give no reference. 

\textbf{Response:}
Added an explanation and relevant references that justify finite-difference methods:
\begin{quote}
Firstly, it should be noted that there exist many methods of numerically approximating the solutions to a system of (ordinary or partial) differential equations; but, finite-difference methods, such as Euler or Runge-Kutta methods, are particularly well suited to solving systems of ODEs [76, 77]. In the problem at hand, since the system of ODEs satisfies the Lipschitz condition of continuity at the lower boundary condition $\underline{\hat{b}}$, if $\underline{\hat{b}}$ was known, standard finite-difference methods would apply [78] (see Section B.2.2, Appendix B for the definition of Lipschitz condition). However, this is not true in both scenarios, the one considered in the literature and the one at hand: the common lower bound on bids, $\underline{\hat{b}}$, is unknown a priori [75].
\end{quote}

\subsection{Comment 22}
Why do the FSM and PPM methods suit your problem? Why select FSM? Need to explain how problem naturally fits into FSM. Justify.

\textbf{Response:}
It was made clear in the thesis why the problem naturally fits into the framework of the FSM method. Furthermore, the particular choice of the FSM and PPM is argumented as well:
\begin{quote}
  The problem naturally fits into the framework of the FSM method since, as discussed in the previous paragraph, if the lower boundary condition, $\underline{\hat{b}}$, was known, standard finite-difference methods could be used to a great success in finding a numerical solution to the system of ODEs. As demonstrated in the subsequent section, the FSM method tries iteratively to guess the lower boundary condition, and for each guess, it then uses finite-difference methods to numerically solve the system of ODEs. Furthermore, the thesis focuses on the FSM and PPM methods since they are the simplest out of all of the available algorithms described in Hubbard and Parsch~[75], and hence, they pose the least technical difficulties when adapting to the problem at hand, and yet yield numerical results of acceptable quality to permit conlusions to be drawn.
\end{quote}

\subsection{Comment 23}
Clarify that EFSM is my own work. Need to differentiate this work from Lebrun.

\textbf{Response:}
A sentenced highlighting the fact that EFSM method was developed by the author was added to the introductory paragraph of the EFSM method:
\begin{quote}
To the best of the author's knowledge, the EFSM method developed in this thesis is the only numerical algorithm that considers all nontrivial equilibria to the system of ODEs in Equation~(5.27) with lower and upper boundary conditions in Equations~(5.28)~and~(5.29) respectively.
\end{quote}

\subsection{Comment 24}
A general point for discussion at the viva concerns with the value of the closed form solutions obtained and their applicability in an operational setting. To what extent do the assumptions made to yield these solutions compromise their operational value. This aspect is further compounded as you move to more numerical solutions.

\textbf{Response:}
In Chapter 5, an additional simplifying assumption is made about the DMP auction: costs for the network operators are uniformly distributed. The impact of this impact on the applicability of the solutions to real life is now discussed in the summary section of Chapter 5:
\begin{quote}
In this chapter, the bidding problem with symmetric cost distributions, as defined in Chapter~4, was transformed into a bidding problem with asymmetric cost distributions. Following the transformation, the equilibrium bidding strategies for the generic case of $n$ network operators were formally characterised; that is, it was shown that the pure strategy Bayesian Nash equilibrium exists and is unique (Proposition~5.3 and Corollary~5.4). This is an important result as it proves that the DMP network selection mechanism is economically well-behaved since the equilibrium exists.

When restricted to $n=2$ network operators, the equilibrium bidding strategies were analytically derived. To aid in the derivation, it was necessary to assume that costs for the network operators were uniformly distributed. Given the lack of knowledge of the way the costs are distributed, it is standard practice to assume the probability of each cost to be uniform~[84]. Nonetheless, such an assumption is limiting and it is highly likely it will not be representative of the reality. Furthermore, in the same case of $n=2$ network operators, the expected prices for the subscriber were analysed, and it was shown that, for any expected price, as the difference between the reputation ratings of the network operators increases, the price weight has to increase (or remain constant) in order to keep the expected price fixed. This observation carries very serious implications on the operation of the DMP, as the subscriber is effectively given the ability to influence the expected prices by an appropriate choice of the price weight.

Finally, for the case of $n\geq 2$ network operators, three numerical algorithms for approximating the equilibrium bidding strategies were proposed: FSM (Algorithm~5.2), PPM (Algorithm~5.3), and EFSM (Algorithm~5.4). When developing the algorithms, similarly to the restricted case with $n=2$ network operators, it was assumed that costs for the network operators were uniformly distributed. Therefore, the same limitations apply. However, generalising algorithms to nonuniform distributions should not prove difficult. The algorithms were further verified for correct implementation, and, for each approximated scenario, the derived equilibrium bidding strategies were tested for sufficiency condition for a pure strategy Bayesian Nash equilibrium. The FSM and PPM methods allow for numerically approximating equilibrium bidding strategies for a subset of all possible bidding scenarios resulting in nontrivial equilibria, while the EFSM method enables computation of the numerical solution to all bidding scenarios. Since, as shown in Section~5.2, analytical derivation of the equilibrium bidding strategies in the case of more than 2 network operators is not possible, the existence of algorithms capable of numerically approximating the solutions is a major step forward in understanding the DMP auction. In fact, the algorithms constitute a tool that network operators participating in the DMP can use to formulate their own bidding strategies and understand the bidding behaviours of other network operators.
\end{quote}

\clearpage

%% Chapter 6
\section{Chapter 6 Comments}
\subsection{Comment 1}
P99: Eq.~6.5 notation issue.\\[-2ex]

\textbf{Response:}
Made appropriate changes in the captions of figures 6.5 and 6.6.

\subsection{Comment 2}
P100: Last equation. Why is denominator 4? 95\% rule. Mention this first.\\[-2ex]

\textbf{Response:}
The paragraph explaining the choice of the parameters was rephrased:
\begin{quote}
Secondly, the distribution specific parameters (mean and standard deviation) need to be specified for each bidder. The choice of the parameters is motivated by the shape of the normal distribution. Therefore, the midpoints of the original supports are picked as means for both bidders, that is,
\begin{equation}
\mu_i = \underline{c}_i + \frac{\bar{c}_i - \underline{c}_i}{2} = \underline{c}_i + \frac{w}{2}.
\end{equation}
Furthermore, noting that, in the case of normal distribution, 95\% of all the values falls within 2 standard deviations away from the mean~[82], the standard deviations are selected to be equal to the quarter of the length of the original supports, that is,
\begin{equation}
\sigma_i = \frac{\bar{c}_i - \underline{c}_i}{4} = \frac{w}{4}.
\end{equation}
In this way, for each bidder, 95\% of all the costs falls within the interval $[\underline{c}_i, \bar{c}_i]$, and therefore, the probability of drawing cost outside this interval is minimised. With this choice of parameters, the truncated normal distributions are effectively imitating uniform distributions with support $[\underline{c}_i, \bar{c}_i]$ for each bidder.
\end{quote}

\subsection{Comment 3}
What is the point of this chapter? What is the motivation? Need to explain how this chapter will be a contribution over and about the previous chapter.

\textbf{Response:}
The introduction to Chapter 6 was rewritten and features the problem of instability and its possible resolutions explained:
\begin{quote}
This chapter presents a methodology for approximating the DMP network selection mechanism with an asymmetric FPA auction with common prior. It is further argued that this methodology constitutes a possible resolution to the potential problem of numerical instability of the FSM and EFSM methods.

Fibich and Gavish~[85] showed that the FSM method and its derivatives, such as the EFSM method, become numerically unstable for large numbers of bidders. The issue has not impacted the results thus far presented in this thesis due to the fact that only the scenarios with as many as 4 network operators were considered. However, it is important to acknowledge the fact that the issue exists and, sooner or later, for large number of network operators, it will affect the numerical solutions generated by FSM and, more importantly, EFSM methods. Therefore, it is vital to address the issue on a proactive rather than reactive basis.

The most obvious way of addressing the issue would be to employ a different numerical method in place of the EFSM method. However, to the best of author's knowledge, the EFSM method is the only numerical algorithm in existence that considers all nontrivial equilibria to the system of ODEs in Equation~(5.27) with lower and upper boundary conditions in Equations~(5.28)~and~(5.29) respectively. Furthermore, it is not immediately obivious how the EFSM method would have to be modified to be based entirely on methods that are not FSM derivative, and hence, do not possess numerical instability issues.

In this chapter, an alternative approach is presented. It is explored whether an auction format represented by the DMP network selection mechanism can be modelled as an asymmetric FPA auction with common prior (henceforth, referred to as CP auction). In a CP auction, the range the costs can vary is the same for each bidder. More formally, the cost distributions for each bidder share the same support. By modelling the DMP network selection mechanism as a CP auction, the numerical solution methods (other than the FSM-based methods) presented in Hubbard and Parsch~[75], and extensively studied by the economic community, could be used to approximate the solution to the DMP auction. This would allow network operators to consider a simpler bidding problem for which there are many well-defined numerical solutions. As a result, presented with a DMP auction, network operators could bid according to the equilibrium bidding strategies of the corresponding CP auction while approximately retaining the expected utility if bidding according to the equilibrium bidding strategies of the DMP auction, and hence, avoiding the need to use the EFSM method to solve the DMP auction.
\end{quote}

\subsection{Comment 4}
Have you had any problems with convergence of the numerical methods? Need to mention that FSM/EFSM has instability.

\textbf{Response:}
The introduction to Chapter 6 now makes it clear that FSM and EFSM may become unstable for large numbers of bidders:
\begin{quote}
Fibich and Gavish~[85] showed that the FSM method and its derivatives, such as the EFSM method, become numerically unstable for large numbers of bidders. The issue has not impacted the results presented in this thesis thus far due to the fact that only the scenarios with as many as 4 network operators were considered.
\end{quote}

\subsection{Comment 5}
Why choose a CP that is a truncated normal distribution?

\textbf{Response:}
The motivation for truncated normal distribution is now explained in the thesis:
\begin{quote}
All that remains is to then select a family of distributions which captures the numerical ranges of the original supports as closely as possible. To provide an illustrative example, let there be 2 bidders such that $\underline{c}_1 < \underline{c}_2 < \bar{c}_1 < \bar{c}_2$. Each bidder is characterised by a uniform distribution. The common support in this case equals $[\underline{c}, \bar{c}] = [\underline{c}_1, \bar{c}_2]$. Firstly, recall that the chosen distirbutions have to satisfy Assumptions~6.1. Thus, uniform distributions considered over the common support cannot be chosen since they violate those assumptions. To see this, let $F_1$ be the cumulative distribution function (cdf) of the uniform distribution with the support $[\underline{c}_1, \bar{c}_1]$. Extended to the common support $[\underline{c}, \bar{c}]$, the derivative of $F_1$, the probability density function (pdf), is zero over the interval $[\bar{c}_1, \bar{c}] = [\bar{c}_1, \bar{c}_2]$, and hence, it is not locally bounded away from zero over the common support. As a result, it is necessary to choose distributions such that they satisfy Assumptions~6.1, and at the same time, possess the shape characteristics similar to the uniform distribution, such as symmetry. One possible way of casting this scenario into common prior setting is to model the distributions of both bidders as truncated normal distributions truncated to the interval $[\underline{c}_1, \bar{c}_2]$, and with differing mean and standard deviation parameters. This is depicted in Figure~6.3.
\end{quote}

\subsection{Comment 6}
P91: What is the physical interpretation of common priors in this application? Does the common priors assumption have any physical interpretation in terms of network selection?

\textbf{Response:}
Common priors assumption is used to address the instability issues of the EFSM algorithm. That is, it is used to approximate the DMP auction so that the EFSM algorithm does not have to be employ in deriving approximate equilibrium bidding strategies. However, this in itself might lead to a game between the network operators as now elaborated upon in the thesis:
\begin{quote}
Modelling of the DMP auction as a CP auction assumes that the network operators will use the equilibrium bidding strategies of the CP auction (CP strategies) as bidding strategies in the DMP auction. However, by Proposition~5.3, the CP strategies do not constitute an equilibrium to the DMP auction; they are merely used as \emph{approximations} to the actual equilibrium bidding strategies of the DMP auction (equilibrium strategies). Therefore, there exists possibility that a network operator might exploit this fact by bidding according to the equilibrium strategies while other network operators will bid according to the CP strategies. Concurrently, however, since the equilibrium strategies can only be derived using the EFSM method, it is likely that the derivation might fail due to the numerical instability of the algorithm. All in all, each network operator faces a tradeoff: bid according to the equilibrium strategies but risk lack of convergence, or bid according to CP strategies but risk other network operators bidding according to the equilibrium strategies. Of course, the magnitude of the problem decreases dramatically as the number of network operators involved in the DMP increases. For then the numerical instability will lead to the divergence of the EFSM algorithm and render the derivation of the equilibrium strategies impossible. Hence, the network operators will be forced to rely on the CP strategies.
\end{quote}

\clearpage

%% Chapter 7
\section{Chapter 7 Comments}
\subsection{Comment 1}
How credible is it to expect network operators to adopt the work of your thesis? Could these techniques be implemented in a practical setting?

\textbf{Response:}
FIX:ME

\subsection{Comment 2}
The market is never in equilibrium---what does this mean for your work?

\textbf{Response:}
FIX:ME

\subsection{Comment 3}
A range of assumptions have been necessary to gain closed-form solutions (and numerical solutions) but this also narrowed the breadth of application---to what extent does this impact on the usefulness of the generalised result you derive?

\textbf{Response:}
FIX:ME

\subsection{Comment 4}
The suggestions for further work are good but are too brief and take no real account of the practical aspect of the work. Further work should explain how results could be broadened and used in practice because this is where the value of the work lies.

\textbf{Response:}
FIX:ME

\end{document} 
