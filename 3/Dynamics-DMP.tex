%!TEX root = /Users/jakubkonka/Thesis/Thesis.tex
\chapter{Dynamics of Network Selection in the Digital Marketplace} % (fold)
\label{cha:dynamics_of_network_selection_in_the_digital_marketplace}

\minitoc
\vspace{10mm}

The results presented in the previous chapter describe only a single stage of the selection process employed in the Digital Marketplace (DMP); i.e., a one-shot DMP auction. As such, the temporal (or dynamic) aspect of the process was neglected. The main reason for such a decision was as follows. Concentrating only on a single stage of the process considerably simplified the inherent analytical complexity of the problem and rendered the analysis tractable in case of two competing bidders. As a result, light was shed on the appropriateness of the DMP auction as a network selection mechanism at least in case of two bidders. In other words, if the results of the analysis of the one-shot DMP auction were unfavorable, then that would already render the mechanism inappropriate, and hence, little would be gained from the inclusion of the temporal aspect in the analysis.

On the other hand, it is important to consider the temporal aspect of the selection process since network selection evolves over time. For example, parameters such as reputation of each network operator, number of subscribers (and price weights), etc., which were considered static thus far, will change over time since the reputation of each network operator depends on their past (service provision) history, and more than one subscriber will enter the marketplace at any one time (and hence, price weights will vary over time).

\section{Mathematical Description} % (fold)
\label{sec:mathematical_description_dynamic}
The conceptual description of the dynamics of network selection mechanism in the DMP is as follows. The subscribers post service requests, and the network operators compete over them. The entire exchange is implemented as a procurement first-price sealed-bid (FPA) auction. The auctions operate in real time, and transaction volumes and prices reflect the relative status of supply and demand. In order to simplify the modeling, it is assumed that the probability of two auctions occurring at the same instance of time is zero. Therefore, the network operators cannot bid at two auctions simultaneously. The subscribers generate a stream of service requests, characterized by a price weight and additional attributes, according to some predetermined probability distribution function. The service request attributes are currently considered to include service type, and required bit-rate.

Since the service requests are generated over time, the network operators repeatadly compete with each other in a series of FPA auctions. Thus, from the game theoretic perspective, the network selection mechanism is akin to a repeated game where each consecutive auction constitutes a stage game~\cite{BasarOlsder1999, Gibbons92}. If we further assume that the costs of network operators may vary over time (i.e., the costs may change with each consecutive service request), then the game becomes a repeated game of incomplete information. As we will demonstrate below, the analytical analysis of this game in search of an (optimal) equilibrium bidding behavior of the network operators is intractable. Therefore, a more experimental approach will be taken; that is, the problem will be analyzed using simulation modeling approach.

In what follows, a rigorous description of the network selection process in the DMP is loosely based on the work of Figliozzi \emph{et al.}~\cite{FigliozziJaillet2008}. Although Figliozzi's work covers the dynamic analysis of a transportation marketplace, it bears many similarities to the operation of the DMP. Let $\{s_1,\ldots,s_M\}$ be the set of arriving service requests such that $s_i = (s^w_i, s^{attr}_i)$ for all $i\in I$, where $I=\{1,\ldots,M\}$ is the set of indices, $s^w_i\in [0,1]$ is the subscriber's price weight, and $s^{attr}_i\in S^{attr}$ is a $k$-tuple of attributes describing the service request. The set of all service request attribute $k$-tuples, $S^{attr}$, is left open-ended; however, at a bare minimum, it should consist of pairs ($2$-tuples) of service type and required bit-rate. For example, the web-browsing service requiring $512$ kbps bit-rate would be described by a pair $s_i^{attr} = (s_i^{type},s_i^{bw}) = (\text{``web''}, 512)$, while the email service requiring 256 kbps by a pair $(\text{``email''}, 256)$.

Let the service request/auction arrival epochs be $\{t_1,\ldots,t_M\}$ such that $t_i < t_{i+1}$ for all $i\in I$, and $t_i$ represents the arrival time of service request $s_i$ for all $i$. It is assumed that there is a one-to-one correspondence between $t_i$ and $s_i$ for all $i$. Furthermore, the service requests and their arrival times are not known in advance; i.e., they are assumed to come from some probability distribution with outcomes $\{\omega_1,\ldots,\omega_M\}$ such that $\omega_i = (t_i, s_i)$ for all $i$.

Following the notation from the previous chapters, let $N$ denote the set of network operators such that $|N| = n < \infty$; that is, there are $n$ network operators competing in the DMP. At any stage $i$, that is at an auction for service request $s_i$, each network operator $j\in N$ submits a bid $b^j_i\in\mathbb{R}_+$. It is further assumed that at each stage, every network operator must participate in an auction. Let $z^j_i\in Z$ denote the available bandwidth of network operator $j$ at time instant $i$ for all $i\in I$ and $j\in N$. Thus, the cost of servicing request $s_i$ by network operator $j$ with available bandwidth $z^j_i$ is denoted by $c^j_i = c^j_i(s_i,z^j_i)$. The cost is assumed private knowledge and, without loss of generality, it is further assumed $c^j_i\in [0,1]$ for all $i,j$.

Furthermore, at each stage $i$, each network operator $j$ is characterized by a reputation rating $r^j_i$ such that $r^j_i\in[0,1]$ for all $i,j$. As in the previous chapter, the reputation ratings are assumed common knowledge. It is further assumed they vary over time, thus reflecting the credibility of network operators in dealing with the subscribers' requests. To this end, each subscriber is assumed to have a binary perception of quality of service; that is, they may either be satisfied or dissatisfied with the service offered by a network operator. More formally, let $\sigma^j_i$ represent the satisfaction of the subscriber requesting the service $s_i$ which is handled by the network operator $j$ such that, for all $i,j$,
\begin{equation}
  \label{eq:def_users_satisfaction_dynamic}
  \sigma^j_i = \left\{
  \begin{array}{ll}
    1 &\text{if satisfied},\\
    0 &\text{otherwise}.
  \end{array}\right.
\end{equation}
It is further assumed that the subscriber is satisfied with the service if the available (at the time of service request) bandwidth of the serving network operator exceeds the required bit-rate of the service; that is,
\begin{equation}
  \label{eq:def_users_satisfaction_bandwidth_dynamic}
  \sigma^j_i = \left\{
  \begin{array}{ll}
    1 &\text{if } s_i^{bw} \le z_i^j,\\
    0 &\text{otherwise}.
  \end{array}\right.
\end{equation}
The motivation for such a simple model of the subscriber's satisfaction is as follows: since it is difficult to quantify how the subscribers perceive satisfaction with the service, rather than experimenting with probabilistic distribution of subscribers' satisfaction, we keep it simple and concentrate on more technical aspects of the DMP such as the bidding behavior of the network operators, or the reputation rating update system.

Since any one service request $i$ can be handled by a single network operator $j$, each network operator will be generating a binary sequence of user satisfaction reports over time; that is, a sequence $(\sigma^j_i)_{i\in I,j\in N}$. In turn, this sequence will then be used to compute and update the network operator's reputation rating. Currently, an adaptation of the reputation update formula proposed by Le Bodic \emph{et al.}~\cite{DMLeBodic00} is assumed; that is,
\begin{equation}
  \label{eq:def_rep_update_lebodic_dynamic}
  r^j_{i+1} = \frac{\displaystyle\sum_{k=i-(d-1)}^{i}{(1-\sigma^j_k)}}{d} \quad\text{for all }i=d,\ldots,M \text{ and }j\in N,
\end{equation}
where $d\in \mathbb{Z}_+$ is the window size. Since $\sigma^j_i$ is only defined for $i\ge 1$, it is further assumed that
\begin{equation*}
  r^j_k = 0.5 \quad\text{for all }k=1,\ldots,d-1\text{ and }j\in N.
\end{equation*}
Therefore, $r^j_k=0.5$ for all $k=1,\ldots,d-1$ can be perceived as the initial reputation rating assigned to each network operator. It is chosen to be the midpoint of the feasible reputation rating interval $[0,1]$ for all network operators so that the reputation rating system is not biased towards any one network operator from the start.

Let $\mathds{1}^j_i$ be an indicator variable for network operator $j$ when service request $s_i$ is being auctioned such that
\begin{equation}
  \label{eq:indicator_function_dynamic}
  \mathds{1}^j_i = \left\{
  \begin{array}{ll}
    1 &\text{if } \beta(s^w_i,b^j_i,r^j_i) < \displaystyle\min_{j'\neq j} \beta(s^w_i,b^{j'}_i,r^{j'}_i),\\
    0 &\text{otherwise},
  \end{array}\right.
\end{equation}
where $\beta$ is defined as in Equation~\eqref{eq:def_beta_static}; that is,
\begin{equation}
  \label{eq:def_beta_dynamic}
  \beta(s^w_i,b^j_i,r^j_i) = s^w_i b^j_i + (1-s^w_i) r^j_i \quad\text{for all } i=1,\ldots,M \text{ and }j\in N.
\end{equation}
In other words, $\mathds{1}^j_i = 1$ if network operator $j$ secures the service request $s_i$, and $\mathds{1}^j_i = 0$ otherwise. The set of indicator variables at each stage $i$ is denoted by $\mathds{1}^N_i=\{\mathds{1}^j_i\}_{j\in N}$ such that $\sum_{j\in N}\mathds{1}^j_i\le 1$; that is, only one network operator can be the winner of an auction at any stage $i$. The profit obtained by network operator $j$ for service request $s_i$ can therefore be expressed as
\begin{equation}
  \label{eq:def_no_profit_dynamic}
  u^j_i = \left(b^j_i - c^j_i\right)\mathds{1}^j_i = \left(b^j_i - c^j_i(s_i, z^j_i)\right)\mathds{1}^j_i \quad\text{for all }i,j.
\end{equation}

Note that when considering a single-stage auction game (see Chapter~\ref{cha:network_selection_mechanism_in_the_digital_marketplace}), the solution to a bidding behavior problem was obtained only in a restricted setting; that is, the number of network operators was restricted to two, and costs were drawn from a uniform distribution over the interval $[0,1]$. The problem was intractable for more than two competing network operators due to the fact that the network operators were asymmetrical. With the added layer of complexity of repeated interactions between the network operators, the problem becomes intractable even in the case of only two network operators, and uniformly distributed costs over the interval $[0,1]$. As described above, the cost of a network operator may vary over time; hence, the problem can be treated as a problem of sequential auctions with bidders with multiunit demand/supply curves, and as explained by Krishna\cite{Krishna10} and Figliozzi~\cite{FigliozziJaillet2008} this class of problems remains intractable.
% section mathematical_description (end)

\section{Simulation Modeling} % (fold)
\label{sec:simulation_modeling_dynamic}
Describe Py3k model of the DMP. Mention that the simulation is nonterminating.

The mathematical description of the problem was adapted and translated into a discrete-event system (DES) simulation model and implemented in Python programming language, version 3 (Py3k). Py3k was chosen mainly due to the fact that Py3k is a dynamically-typed language which makes prototyping quicker than in a statically-typed language such as C/C++. This allows the researcher to focus more on the design and results of the experiment rather than its implementation. Furthermore, Py3k features a set of libraries that further simplify the implementation stage of the DES simulation; for example, `subprocess' module allows to execute multiple simulation runs in parallel~\cite{Py3kSubprocess}, while `numpy', `scipy', and `matplotlib' modules provide a set of convenience routines for numerical computation and scientific plotting in Py3k~\cite{Numpy, Scipy, Matplotlib}.
% section simulation_modeling (end)

\section{Simulation Analysis Toolbox} % (fold)
\label{sec:simulation_analysis_toolbox_dynamic}
Describe the simulation workflow, including the analysis of transient state; that is, how I go about removing the transient response while searching for the steady-state of a parameter.
% section simulation_analysis_toolbox (end)

\section{Results and Analysis} % (fold)
\label{sec:results_and_analysis_dynamic}
Analyze reputation rating update systems proposed by Alisdair and Le Bodic. Compare and contrast. Show that they are initial conditions sensitive.
% section results_and_analysis (end)
% chapter dynamics_of_network_selection_in_the_digital_marketplace (end)
