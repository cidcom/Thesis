%!TEX root = /Users/jakubkonka/Thesis/Thesis.tex
\chapter{Dynamics of Network Selection in the Digital Marketplace} % (fold)
\label{cha:dynamics_of_network_selection_in_the_digital_marketplace}

\minitoc
\vspace{10mm}

The results presented in the previous chapter describe only a single stage of the selection process employed in the Digital Marketplace (DM); i.e., a one-shot DM auction. As such, the temporal (or dynamic) aspect of the process was neglected. The main reason for such a decision was as follows. Concentrating only on a single stage of the process considerably simplified the inherent analytical complexity of the problem and rendered the analysis tractable in the case of two competing bidders. As a result, light was shed on the appropriateness of the DM auction as a network selection mechanism at least in the case of two bidders. In other words, if the results of the analysis of the one-shot DM auction were unfavourable, then that would already render the mechanism inappropriate, and hence, little would be gained from the inclusion of the temporal aspect in the analysis.

On the other hand, it is important to consider the temporal aspect of the selection process since network selection evolves with time. For example, parameters such as reputation of each bidder, number of buyers (and price weights), etc., which were considered static thus far, will change with time since the reputation of each bidder depends on their past (service provision) history, and more than one buyer will enter the marketplace at any one time (and hence, price weights will vary with time).

The conceptual description of the network selection process in the DM is as follows. The end-users post service requests, and the network operators compete over them. The entire exchange is implemented as a procurement first-price sealed-bid (FPA) auction. The auctions operate in real time, and transaction volumes and prices reflect the relative status of supply and demand. In order to simplify the modelling, it is assumed that the probability of two auctions occurring at the same instance of time is zero. Therefore, the network operators cannot bid at two auctions simultaneously. The end-users generate a stream of service requests, characterised by price weight and additional attributes, according to some predetermined probability distribution function. The service request attributes are currently considered to include service type, and required bit-rate.

In what follows, a rigorous description of the network selection process in the DM is loosely based on the work of Figliozzi \emph{et al.}~\cite{FigliozziJaillet2008}. Although Figliozzi's work covers the dynamic analysis of a transportation marketplace, it bears many similarities to the operation of the DM. Let $\{s_1,\ldots,s_M\}$ be the set of arriving service requests such that $s_i = (s^w_i, s^{attr}_i)$ for all $i=1,\ldots,M-1$, where $s^w_i\in [0,1]$ is the buyer's price weight, and $s^{attr}_i\in S^{attr}$ is a $k$-tuple of attributes describing the service request. The set of all service request attribute $k$-tuples, $S^{attr}$, is left open-ended; however, at a bare minimum, it should consist of pairs ($2$-tuples) of service type and required bit-rate. For example, the web-browsing service requiring $512$ kbps bit-rate would be described by a pair $(\text{``web''}, 512)$, while the email service requiring 256 kbps by a pair $(\text{``email''}, 256)$.

Let the service request/auction arrival epochs be $\{t_1,\ldots,t_M\}$ such that $t_i < t_{i+1}$ for all $i$, and $t_i$ represents the arrival time of service request $s_i$ for all $i$. It is assumed that there is a one-to-one correspondence between $t_i$ and $s_i$ for all $i$. Furthermore, the service requests and their arrival times are not known in advance; i.e., they are assumed to come from some probability distribution with outcomes $\{\omega_1,\ldots,\omega_M\}$ such that $\omega_i = (t_i, s_i)$ for all $i$.

Following the notation from the previous chapters, let $N$ denote the set of bidders such that $|N| = n < \infty$; that is, there are $n$ bidders competing in the DM. At any stage $i$, that is at an auction for service request $s_i$, each bidder $j\in N$ submits a bid $b^j_i\in\mathbb{R}_+$. It is further assumed that at each stage, every bidder must participate in an auction. Let $z^j_i\in Z$ denote the available bandwidth of bidder $j$ at time instant $i$ for all $i=1,\ldots,M-1,\: j\in N$. Thus, the cost of servicing request $s_i$ by network operator $j$ with available bandwidth $z^j_i$ is denoted by $c^j_i = c^j_i(s_i,z^j_i)$. The cost is assumed private knowledge and, without loss of generality, it is further assumed $c^j_i\in [0,1]$ for all $i,j$. Let $\mathds{1}^j_i$ be the indicator function for bidder $j$ for service request $s_i$ such that $\mathds{1}^j_i=1$ if bidder $j$ won the auction for service request $s_i$, and $\mathds{1}^j_i=0$ otherwise. The set of indicator variables at each stage $i$ is denoted by $\mathds{1}^N_i=\{\mathds{1}^j_i\}_{j\in N}$ such that $\sum_{j\in N}\mathds{1}^j_i\le 1$; that is, only one bidder can be the winner of an auction at any stage $i$. The profit obtained by bidder $j$ for service request $s_i$ can therefore be expressed as
\begin{equation*}
  \pi^j_i = \left(b^j_i - c^j_i\right)\mathds{1}^j_i = \left(b^j_i - c^j_i(s_i, z^j_i)\right)\mathds{1}^j_i
\end{equation*}
for all $i,j$.

After each auction, a set of publicly observed information, $Y_i$ for all $i$, is revealed.

Perhaps, any remaining sections will tackle different (simplifying) assumptions: costs drawn only once, etc. Are all assumptions related to different information disclosure cases?

% chapter dynamics_of_network_selection_in_the_digital_marketplace (end)
