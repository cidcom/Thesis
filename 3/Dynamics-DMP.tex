%!TEX root = /Users/jakubkonka/Thesis/Thesis.tex
\chapter{Dynamics of Network Selection in the Digital Marketplace} % (fold)
\label{cha:dynamics_of_network_selection_in_the_digital_marketplace}

\minitoc
\vspace{10mm}

The results presented in the previous chapter describe only a single stage of the selection process employed in the Digital Marketplace (DM); i.e., a one-shot DM auction. As such, the temporal (or dynamic) aspect of the process was neglected. The main reason for such a decision was as follows. Concentrating only on a single stage of the process considerably simplified the inherent analytical complexity of the problem and rendered the analysis tractable in the case of two competing bidders. As a result, light was shed on the appropriateness of the DM auction as a network selection mechanism at least in the case of two bidders. In other words, if the results of the analysis of the one-shot DM auction were unfavourable, then that would already render the mechanism inappropriate, and hence, little would be gained from the inclusion of the temporal aspect in the analysis.

On the other hand, it is important to consider temporal aspect of the selection process since network selection evolves with time. For example, parameters such as reputation of each bidder, only one buyer (and hence, constant price weight), etc., which were considered static thus far, will change with time since the reputation of each bidder depends on their past (service provision) history, and more than one buyer will enter the marketplace at any one time (and hence, price weight will vary with time).

Describe the model (mathematically) here...

Proceed to statement that the game is actually a repeated game...

Describe different possible information disclosure models and their possible implications...

% chapter dynamics_of_network_selection_in_the_digital_marketplace (end)
