\chapter*{Abstract} % (fold)
\label{cha:abstract}
\addcontentsline{toc}{chapter}{Abstract}

Mobile communications has become an indispensable part of our everyday lives, with increasingly more people owning a smartphone, and being given access to a plethora of wireless access technologies: WiFi, 3G, and 4G. In an environment of such diversity, where each wireless access technology has its own distinct characteristics, network selection mechanisms provide an efficient way of handling communications services by matching the services' required quality with the characteristics of a particular access technology.

This thesis explores the economic aspects of intelligent network selection in the context of Digital Marketplace---a theoretical market-based framework where network operators compete in a procurement auction-based setting for the right to transport the user's requested service over their infrastructure. It investigates the suitability of a first-price sealed-bid auction as a network selection mechanism. Since this auction-based mechanism constitutes the main trading mechanism of the Digital Marketplace, the results reported herein affect its feasibility as a market for trading wireless communications services of the future. Since it lacks extensive and rigorous economic analysis, this thesis addresses this deficiency by providing an extensive game theoretic analysis of the network selection mechanism.

This thesis creates an economic model of the network selection mechanism, and is the first to characterise the equilibrium bidding behaviour for an arbitrary number of network operators participating in the Digital Marketplace. It proposes three novel numerical methods that allow for numerical derivation of the equilibrium bidding behaviour: forward shooting method (FSM), polynomial projection method (PPM), and extended FSM (EFSM). The FSM and PPM methods allow for numerically approximating equilibrium bidding behaviour for a subset of all possible bidding scenarios, while the EFSM method enables computation of the numerical solution to all bidding scenarios. Finally, since the EFSM method becomes numerically unstable for large number of network operators, a novel methodology for approximating the network selection mechanism with an auction format for which there exist many well-defined and extensively studied numerical solutions is discussed. 
% chapter abstract (end)
