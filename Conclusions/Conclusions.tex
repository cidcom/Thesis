%!TEX root = /Users/jakubkonka/Thesis/Thesis.tex
\chapter{Conclusions and Further Work}
\label{cha:conclusions}

\section{Conclusions} % (fold)
\label{sec:conclusions_conclusions}
The world of mobile communications is becoming increasingly diverse in terms of different wireless access technologies available: WiFi, 3G, and the cutting-edge 4G are gradually being rolled out in many countries across the world. In an environment of such diversity and heterogeneity, where each wireless access technology has its own distinct characteristics, intelligent network selection provides a resource efficient way of handling communications services by matching the services' required quality with the characteristics of a particular access technology.

To make full use of this increasingly diverse environment and increase the competition between network operators even further, the one-to-one mapping between network operators and subscribers need no longer hold. This allows the subscribers to seamlessly switch not only between different wireless access technologies belonging to one particular network operator, but also between network operators themselves. In this way, the subscriber, when requesting a service, is given the option to select a network operator and a wireless access technology that best matches the required quality requirements of the service. It is not only to the benefit of the subscribers, however, since the integration of wireless access technologies will allow network operators for improved revenue generation, and more efficient usage of network resources.

This thesis explored the economic aspects of intelligent network selection. The problem was studied within the context of Digital Marketplace---a \annotate{C3.5}{theoretical} market-based framework \annotate{C1.5}{for trading wireless communications services. It was first proposed by Irvine~\emph{et al.}~in 2000~\cite{DMLeBodic00,DMIrvine00}, and it was developed with the heterogeneous wireless communications environment in mind, where the subscribers have the ability to select a network operator that reflects their preferences on a per service basis.} Since the Digital Marketplace was created with free market (or ``perfect'' competition) in mind, it is particularly well-suited towards the management of future wireless environment where wireless access is traded on a per service basis. It is for this reason that this research explored the problem of network selection within the context of Digital Marketplace.

The network selection mechanism advocated by the Digital Marketplace lacked extensive and rigorous economic analysis. With the game theoretic analysis presented in this thesis, this deficiency has been addressed. More specifically, in Chapter~\ref{cha:direct}, \annotate{C7.3}{a game-theoretic model of the network selection mechanism was formally defined. Several simplifying assumptions were made in order to keep the analysis mathematically tractable. For example, the network operators and the subscriber are risk neutral, and the subscriber does not have any budget constraints. Despite the fact that those assumptions are not entirely representative of the reality, following in the footsteps of von Neumann and Morgenstern, the mathematical theory of an economic phenomenon should be rigorous and developed gradually~\cite{VonNeumann2004}. Therefore, the simplifying assumptions made in this chapter and thesis serve as a starting point for the rigorous, gradual development of the economic theory of operation of the network selection mechanism in the Digital Marketplace, before it can embark on capturing the reality to a high degree.}

Furthermore, in the chapter, the equilibrium bidding strategies were derived for three special/extreme cases. In the first case, when only reputation ratings of the network operators decide on the winning network operator, it was shown that network operators will find it beneficial to submit abnormally high bids, since their bid is independent of the probability of winning the auction. \annotate{C7.3}{While this result sounds like a potential design flaw, in reality, the subscribers will necessarily be budget constrained, and therefore, abnormally high bidding of the network operators will translate into charging the subscribers a premium price for the service that is within the limits of their respective budgets.} In the second case, when only the monetary bids of the network operators matter in the selection of the winner, it was shown that the problem reduces to a standard first-price sealed-bid auction with symmetric bidders, and therefore, the symmetric equilibrium bidding strategies of the standard first-price sealed-bid auction applies. Similarly, the third case, when all network operators are characterised by the same reputation rating, was shown to be a special case of the second case. \annotate{C7.3}{In both cases, the abundance of theoretical results and economic insight from the auction literature applies, found, for example, in Krishna~\cite{Krishna10}.}

Finally, the equilibrium bidding strategies for only two network operators were analytically derived. It was shown that although the derived equilibium bidding strategies allows for negative bids, it does not lead to negative profit in case of winning (or a tie) of either network operator. In fact, it was established that the direct mechanism representation of the Digital Marketplace auction satisfies both individual rationality and incentive compatibility constraints. As a result, this proved that the network operators would find it in their best interest to participate in the auction, and they would reveal their costs truthfully. \annotate{C7.3}{It was further noted that the real behaviour of the network operators might be dictated by the need to secure the contract with the subscriber first and foremost, and hence, lead to negative bidding; a strategy akin to the ``loss leader'' pricing strategy. However, since the ultimate aim of this thesis was gradual development of rigorous economic theory of the operation of the network selection mechanism within the Digital Marketplace, network operators were always assumed to behave rationally from the perspective of game theory, and bid at least their cost.}

In Chapter~\ref{cha:indirect}, by mathematically transforming the problem into an alternate form, it was shown that the equilibrium bidding strategies exist and are unique. \annotate{C7.3}{This was an important result from the perspective of the development of rigorous theory of the operation of the network selection mechanism as it proved that the mechanism is economically well-behaved since the equilibrium exists.} Furthermore, the equilibrium bidding strategies were explicitly derived in the case of two network operators and their costs assumed to be uniformly distributed. \annotate{C7.3}{The assumption of uniform distribution of costs for the network operators was argumented by the fact that it is a standard practice when there is a lack of knowledge of the actual type of the distributions~\cite{Chung2004}. Nevertheless, it was noted that such an assumption is limiting, and it is highly likely it will not be fully representative of the reality.} Furthermore, in the case of two network operators, the expected prices the subscribers will have to pay for different values of the price weight were examined. \annotate{C7.3}{It was shown that, for any expected price, as the difference between the reputation ratings of the network operators increases, the price weight has to increase (or remain constant) in order to keep the expected price fixed. It was noted that this observation carries very serious implications on the operation of the Digital Marketplace, as the subscriber is effectively given the ability to influence the expected prices by an appropriate choice of the price weight.}

Finally, three numerical methods for numerically approximating the equilibrium bidding strategies in the case of more than two network operators were proposed: forward shooting method (FSM), polynomial projection method (PPM), and extended FSM (EFSM). \annotate{C7.3}{When developing the algorithms, similarly to the restricted case with two network operators, it was assumed that costs for the network operators were uniformly distributed; therefore, the same limitations applied. However, generalising the algorithms to nonuniform distributions should not prove difficult since other researchers have successfully employed similar numerical methods for studying problems where distributions were nonuniform~\cite{HubbardPaarsch2011}.} The FSM and PPM methods allow for numerically approximating equilibrium bidding strategies for a subset of all possible bidding scenarios resulting in nontrivial equilibria, while the EFSM method enables computation of the numerical solution to all bidding scenarios. It should be noted at this point that the development of the EFSM method constitutes an indirect contribution of this thesis. The method allows for approximating solution to first-price sealed-bid auction with asymmetric bidders posed by the Digital Marketplace bidding problem, and to the best of the author's knowledge, this type of auctions has not yet been solved numerically by the economic community. \annotate{C7.3}{Since the analytical derivation of the equilibrium bidding strategies in the case of more than two network operators is not possible, the existence of algorithms capable of numerically approximating the solutions is a major step forward in the development of the economic theory of operation of the network selection mechanism in the Digital Marketplace. Furthermore, the algorithms constitute a tool that the network operators participating in the Digital Marketplace can use to formulate their own bidding strategies, and understand the bidding behaviours of other network operators.}

\annotate{C7.3}{It is a well-known fact that the FSM method and its derivatives, such as the EFSM method, become numerically unstable for large number of bidders~\cite{FibichGavish2011}. While this issue did not impact the results presented in the thesis (due to the fact that only as many as five network operators were considered), it is important to acknowledge the fact that the issue exists. Therefore, in order to address the problem of numerical instability, Chapter~\ref{cha:approximation} explored whether an auction format represented by the network selection mechanism employed in the Digital Marketplace can be modelled as an auction with common prior.} In an auction with common prior, the range the costs can vary is the same for each bidder. To this end, the methodology for casting the bidding problem posed by the Digital Marketplace into the auction with common prior was presented, and the methodology for quantifying the accuracy of the approximation was outlined. Finally, the chapter concluded with the presentation of approximation results for four bidding scenarios with two, three, four and five bidders respectively. It was shown that approximating the network selection mechanism employed by the Digital Marketplace with an auction with common prior consitutes a valid alternative, and as such, even though not perfectly accurate (mean relative errors as large as almost 16\% for all bidders), it might be a more desirable option for the network operators due to the wealth of numerical methods available that have been extensively studied by the researchers.

\annotate{C7.1}{To conclude, the work reported in this thesis constitutes the first step towards the development of rigorous economic theory of the operation of the network selection mechanism in the Digital Marketplace. As such, the participating network operators can use the results derived in this thesis to formulate their pricing strategies and understand the bidding behaviour of their opponents. The subscribers, on the other hand, are in the position to understand the prices they will be required to pay depending on their preferences for the requested service (in terms of the price weight) and the reputation ratings of the participating network operators. However, due to the fact that many simplifying assumptions had to be used in the development of the theory, it will not be fully representative of the reality, and therefore, the results presented should be taken ``with a grain of salt''.}

\annotate{C7.2}{Furthermore, this thesis did not consider the temporal aspect of the network selection mechanism. That is, the game was assumed to be static as opposed to dynamic. In other words, it was assumed that if the network operators were to interact in an auction more than once, they would discount any previous, historic interactions. This is a very strong assumption and it limits the applicability of the results to real-life scenarios for two reasons: 1) it ignores the forces of supply and demand, and hence, the possibility of a market equilibrium, regardless of whether it is actually achievable; and 2) as discussed by Figliozzi~\emph{et al.}, repeated interaction of bidders in an auction will inevitably lead to information revelation, and induce bidding strategies that diverge from the equilibrium bidding strategies derived in this thesis~\cite{FigliozziJaillet2008}.}
% section conclusions (end)

\section{Further Work} % (fold)
\label{sec:further_work_conclusions}
There are many aspects of the research presented in this thesis that can further be elaborated upon. The most important future directions are as follows.
\begin{enumerate}
\item The implementation of the EFSM method assumes the bidders to be characterised by uniform distributions of costs (see Section~\ref{sub:extended_forward_shooting_method_indirect}, Chapter~\ref{cha:indirect}). It will strengthen the generality of the results presented in this thesis if the EFSM method is implemented for bidders characterised by nonuniform distributions as well. Furthermore, it will provide a basic (benchmark) method for numerically solving the unusual first-price sealed-bid auction with asymmetric bidders posed by the Digital Marketplace bidding problem.
\item Currently, the expected prices are examined in the case of only two network operators (see Section~\ref{sub:subscriber_s_perspective_expected_prices_indirect}, Chapter~\ref{cha:indirect}). The subscribers involved in the Digital Marketplace would benefit from characterisation of the expected prices in scenarios with more than two network operators.
\item Furthermore, casting the auction format represented by the network selection mechanism employed in the Digital Marketplace into an auction with common prior assumes that the latter is based on truncated normal distributions (see Section~\ref{sec:network_selection_mechanism_cast_into_common_priors_setting_approximation}, Chapter~\ref{cha:approximation}). However, with the aim of reducing the upper bound on approximation error (16\%), utilisation of different distributions could be explored.
\item Finally, the results presented in this thesis are based on the following fundamental assumptions: subscriber does not have any budget constraints; subscriber and network operators are risk neutral; and network operators are characterised by symmetric uniform cost distributions (see Section~\ref{sec:problem_definition_and_assumptions_direct}, Chapter~\ref{cha:direct}). It is generally agreed that these assumptions do not reflect the real world to a high degree~\cite{Krishna10}. Therefore, relaxing those assumptions and modifying the analysis accordingly might strengthen the applicability of the results to the real world scenarios.
\end{enumerate}
% section further_work (end)
