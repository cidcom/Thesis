\chapter{Introduction} % (fold)
\label{cha:introduction}

\minitoc
\vspace{10mm}

The world of mobile communications is becoming increasingly diverse in terms of different wireless access technologies available: GSM, 3G, WiFi, and the cutting-edge 4G technology, LTE, gradually being rolled out in many countries including the USA \cite{VerizonLTEUSA}, and the UK \cite{BBCLTEUK}. In an environment of such diversity and heterogeneity, where each wireless access technology has its own distinct characteristics, network selection mechanisms provide a resource efficient way of handling communications services by matching the services' required quality with the characteristics of a particular access technology \cite{HossainBeaubrun09}. The importance of these mechanisms is emphasized by the fact that multimode smartphones (iPhones, Android phones, BlackBerry phones) and tablets (iPads, Android tablets) currently dominate the market thus enabling users to connect to many of the available wireless access technologies.

This diversity opens exciting, new possibilities in both the technological and economic sense. The exclusive one-to-one mapping between network operators and subscribers need no longer hold; when requesting a bearer service, the network selection mechanism will be responsible for selecting the network operator (access technology) that best matches the required quality requirements of the service. From the subscribers' perspective, this permits the ability to seamlessly connect at any time, at any place, and to the technology offering the highest quality available for the best price: a paradigm referred to as \emph{Always Best Connected} \cite{ABC03}. From the network operators' perspective, the integration of wireless access technologies will allow for more efficient usage of network resources, and hence, the most economic way of providing both universal coverage, and broadband access \cite{HossainBeaubrun09}.

However, there also exists the possibility of a `tussle' since there are many different actors with opposing interests involved \cite{Clark02}. For example, it is in the best interest of subscribers to obtain the highest quality of the service for the lowest price. Network operators, on the other hand, aim to maximize their profit by performing efficient load balancing. Furthermore, the situation may become even more complex should the service provision be decoupled from the network operators; that is, if the service provision is handled by a separate entity, service provider, while network operators are left with handling of the transport provision \cite{DMBushTussle09}. Therefore, the problem of network selection, which was considered to be technologically difficult, can also be considered to be the problem of economics where wireless access, traded on a per connection basis, is an electronic good that is sold to the subscribers.

\section{Research Objectives} % (fold)
\label{sec:research_objectives_introduction}

% section research_objectives_introduction (end)

\section{Main Contributions} % (fold)
\label{sec:main_contributions_introduction}

% section main_contributions_introduction (end)

\section{Thesis Outline} % (fold)
\label{sec:thesis_outline_introduction}

% section thesis_outline_introduction (end)

% chapter introduction (end)