%!TEX root = /Users/jakubkonka/Thesis/Thesis.tex
\chapter{Introduction} % (fold)
\label{cha:introduction}

\section{Research Objectives} % (fold)
\label{sec:research_objectives_introduction}
Mobile communications has become an indispensable part of our everyday lives. According to Ofcom~\cite{OfcomReport2013}, 51\% of all adults in the UK own a smartphone, and approximately 24\% of all UK households own a tablet. Furthermore, one in five adults declares they would miss their mobile most if it were taken away. It should be noted that these numbers continue to rise, and with each year the penetration of mobile communications will increase.

Parallel to this, mobile users (henceforth referred to as subscribers) are given access to a plethora of wireless access technologies: from WiFi, through 3G, to the latest 4G. Cities throughout the UK are now offering free WiFi hotspots \cite{BBCWiFiGlasgow2014}. Furthermore, according to Ofcom~\cite{OfcomLTE2013}, while 3G already covers 98\% of the UK population indoors, this figure is promised to be at least matched by the 4G mobile services by the end of 2017 at the latest. In an environment of such diversity and heterogeneity, where each wireless access technology has its own distinct characteristics, network selection mechanisms provide an efficient way of handling communications services by matching the services' required quality with the characteristics of a particular access technology \cite{HossainBeaubrun09}. The importance of these mechanisms is emphasized by the fact that multimode smartphones (iPhones, Android phones, BlackBerry phones) and tablets (iPads, Android tablets) currently dominate the market thus enabling subscribers to connect to many of the available wireless access technologies.

This diversity opens exciting, new possibilities in both the technological and economic sense. The exclusive one-to-one mapping between network operators and subscribers need no longer hold; when requesting a bearer service, the network selection mechanism will be responsible for selecting the network operator (access technology) that best matches the required quality requirements of the service. From the subscribers' perspective, this permits the ability to seamlessly connect at any time, at any place, and to the technology offering the highest quality available for the best price: a paradigm referred to as \emph{Always Best Connected} \cite{ABC03}. From the network operators' perspective, the integration of wireless access technologies will allow for more efficient usage of network resources, and hence, the most economic way of providing both universal coverage, and broadband access \cite{HossainBeaubrun09}.

However, there also exists the possibility of a ``tussle'' since there are many different actors with opposing interests involved \cite{Clark02}. For example, it is in the best interest of subscribers to obtain the highest quality of the service for the lowest price. Network operators, on the other hand, aim to maximise their profit by performing efficient load balancing. Furthermore, the situation may become even more complex should the service provision be decoupled from the network operators; that is, if the service provision is handled by a separate entity, service provider, while network operators are left with handling of the transport provision \cite{DMBushTussle09}. Therefore, the problem of network selection, which was considered to be technologically difficult, can also be considered to be the problem of economics where wireless access, traded on a per connection basis, is an electronic good that is sold to the subscribers.

This thesis explores the economic aspects of intelligent network selection. The problem is studied within the context of Digital Marketplace---a market-based framework where network operators compete in a procurement auction-based setting for the right to transport the subscriber's requested service over their infrastructure~\cite{DMLeBodic00}. Since the Digital Marketplace was created with free market (or ``perfect'' competition) in mind, it is particularly well-suited towards the management of future wireless environment where wireless access is traded on a per connection basis. It is for this reason that this research explores the problem of network selection within the context of Digital Marketplace.

More specifically, the main research objectives of the work reported in this thesis are to:
\begin{enumerate}
\item review approaches to intelligent network selection by other researchers;
\item understand the fundamental assumptions about the operation of the Digital Marketplace;
\item create a simple economic model of the network selection mechanism described in the Digital Marketplace;
\item apply game theory, and in particular auction theory, to study the model and derive equilibrium bidding strategies for the network operators.
\end{enumerate}
% section research_objectives_introduction (end)

\section{Main Contributions} % (fold)
\label{sec:main_contributions_introduction}
As briefly outlined in the previous section, in order to fully utilise the wealth and heterogeneity of the future wireless access environment, intelligent network selection is a necessary condition. Without it, subscribers will only be able to achieve a suboptimal ratio of price to quality of the received services, while network operators will struggle to maximise their profit and usage of their resources. Intelligent network selection on its own, however, is not an immediately obvious sufficient condition either. A landscape of such diversity and with so many different actors involved will inevitably lead to a ``tussle''. Therefore, a careful economic analysis of the problem is needed. This thesis strives to address those points through the following main contributions:
\begin{enumerate}
\item This thesis presents an economic analysis of the network selection mechanism in the Digital Marketplace. It should be noted, however, that the analysis and results presented in this thesis are easily extrapolated from the context of the Digital Marketplace. As a matter of fact, one of the main aims of this research was to keep the analysis as generic as possible so that the results can easily be applied elsewhere.
\item Furthermore, this thesis aims at filling the gap in the research on the Digital Marketplace by providing a comprehensive analysis of the network selection mechanism from the economic perspective. Indeed, with the results presented in this thesis, the Digital Marketplace can now be considered a framework of choice for the management of wireless access networks of the future.
\item Finally, this thesis adapts numerical algorithms for approximating first-price sealed-bid auction with asymmetric bidders to the bidding problem posed by the network selection mechanism in the Digital Marketplace. In this case, the asymmetry of the bidders is unusual and no existing numerical methods are directly applicable. This thesis addresses that deficiency by proposing an extented numerical method that tackles the unusual aspect of the network selection in the Digital Marketplace.
\end{enumerate}
% section main_contributions_introduction (end)

\section{Thesis Outline} % (fold)
\label{sec:thesis_outline_introduction}
This thesis is organised as follows. In Chapter~\ref{cha:intelligent}, the concept and importance of intelligent network selection in future wireless access networks is explained. To this end, the chapter outlines the concepts of heterogeneous wireless access network and Always Best Connected paradigm, and outlines the role of network selection. Then, a summary of previous research on intelligent network selection is given. Finally, the contributions of the research presented in this thesis to the problem of intelligent network selection are outlined.

Chapter~\ref{cha:dmp} introduces the concept of Digital Marketplace and its fundamental assumptions. It starts with description of the principles of operation of the Digital Marketplace, and then moves onto a brief overview of auction theory. The overview is necessary to understand the fundamental assumptions of the network selection mechanism employed by the Digital Marketplace which are subsequently outlined. Finally, the chapter concludes with a summary of previous research on Digital Marketplace, and outlines how the research work reported in this thesis complements the work of other researchers.

In Chapter~\ref{cha:direct}, the network selection mechanism employed by the Digital Marketplace is cast into the framework of game theory. The mechanism is then directly analysed in three special cases: 1) when only reputation ratings of the network operators decide on the winning network operator, 2) when only the monetary bids of the network operators matter in the selection of the winner, and finally, 3) when all network operators are characterised by the same reputation rating. In essence, those special cases correspond to the extremes of the studied bidding problem, and in all those cases, the equilibrium bidding strategies are derived. The chapter then concludes with the analysis of the mechanism in a restricted case with only two network operators, for which the equilibrium bidding strategies are derived. They are shown, however, to be suboptimal as they allow the network operators to submit a negative monetary bid.

In Chapter~\ref{cha:indirect}, the studied problem is transformed into a problem that has already been researched by the economic community, and hence, there exist results that are applicable to the problem at hand. The chapter then proceeds to characterising the equilibrium bidding strategies (their existence and uniqueness) in the generic case; that is, with an arbitrary number of network operators and arbitrary distributions of costs. The equilibrium bidding strategies are then explicitly derived in the restricted case; that is, with the number of network operators restricted to two and costs uniformly distributed. Finally, the chapter concludes with the presentation of three numerical methods: forward shooting method (FSM), polynomial projection method (PPM), and extended FSM (EFSM). The methods can be used to numerically approximate the equilibrium bidding strategies in the case of more than two network operators characterised by uniform distributions of costs. The first two of the presented methods, FSM and PPM, allow for numerically approximating equilibrium bidding strategies for a subset of all possible bidding scenarios resulting in nontrivial equilibria, while the third method (EFSM) enables computation of the numerical solution to all bidding scenarios.

The EFSM method becomes numerically unstable for large number of bidders. Therefore, Chapter~\ref{cha:approximation} explores whether an auction format represented by the network selection mechanism employed in the Digital Marketplace can be modelled as an auction with common prior. In an auction with common prior, the range the costs can vary is the same for each bidder. For this type of problem, there are many well-defined and extensively studied numerical solutions. In the first instance, the assumptions governing an auction with common prior are described, and the existence and uniqueness of the equilibrium bidding strategies is formally defined. Following that a numerical method tailored specifically to the auction with common prior is presented. Having derived the numerical method for approximating the equilibrium in the auction with common prior, the methodology for casting the original problem into the auction with common prior is discussed. Finally, the methodology for quantifying the accuracy of the approximation is presented, and the chapter concludes with the presentation of approximation results for four bidding scenarios with two, three, four and five bidders respectively.

Finally, Chapter~\ref{cha:conclusions} draws final conclusions, and discusses future work. Furthermore, in Appendix~\ref{cha:notation}, the mathematical notation used in this thesis is explained, and an overview of the more important mathematical concepts necessary to understand the work reported in this thesis is provided.
% section thesis_outline_introduction (end)

% chapter introduction (end)