%!TEX root = /Users/jakubkonka/Thesis/Thesis.tex
\chapter{Literature Review} % (fold)
\label{cha:literature}

\minitoc
\vspace{10mm}

The world of mobile and wireless communications is becoming increasingly diverse in terms of different wireless access technologies available: in addition to the already established family of technologies (GSM, 3G, and WiFi), the cutting-edge 4G technology, LTE, is gradually being rolled out in many countries including the USA \cite{VerizonLTEUSA}, or is actively being tested for the eventual implementation in the near future in countries such as the UK \cite{BBCLTEUK}. In an environment of such diversity and heterogeneity, where each wireless access technology has its own distinct characteristics, network selection mechanisms provide a more resource efficient way of handling communications services by matching the services' required quality with the characteristics of a particular access technology \cite{HossainBeaubrun09, HossainTalebiFard09}. The importance of such mechanisms is further emphasized by the fact that multimode devices such as smartphones (iPhones, Android phones, BlackBerry phones) and tablets (iPads, Android tablets) currently dominate the market enabling users to connect to many of the available wireless access technologies, including WiFi and 3G.

This diversity combined with the availability of smartphones opens exciting, new possibilities in both the technological and economic sense. The exclusive one-to-one mapping between network operators and subscribers need no longer hold; when requesting a service (e.g., phone call, web browsing, email), the network selection mechanism will be responsible for selecting the network operator and the access technology that best matches the quality requirements of the service. From the subscribers' perspective, this will lead to the ability to seamlessly connect at any time, at any place, and to the technology offering the most optimal quality available for the best price: a paradigm referred to as \emph{Always Best Connected} \cite{ABC03}. From the network operators' perspective, the integration of wireless access technologies will allow for more efficient usage of network resources, and might be the most economic way of providing both universal coverage, and broadband access \cite{HossainBeaubrun09}.

However, there also exists the possibility of a `tussle' since there are many different actors with opposing interests involved \cite{Clark02}. For example, it is in the best interest of subscribers to obtain the highest quality of the service for the lowest price. Network operators, on the other hand, aim to maximize their profit by performing efficient load balancing. Furthermore, the situation may become even more complex should the service provision be separated from the network operators; that is, if the service provision is handled by a separate entity, service provider, while network operators are left with handling of the transport provision \cite{DMBushTussle09}. Therefore, the problem of network selection, which was considered to be technologically difficult, can also be considered to be the problem of economics where communications services, traded on a per service basis, are the electronic goods that are sold to the subscribers.

In this research, we analyse the network selection mechanism proposed in the Digital Marketplace---a market-based framework where network operators compete in a procurement auction-based setting for the right to transport the subscriber's requested service over their infrastructure \cite{DMLeBodic00}. Within this framework, the network selection mechanism is akin to a market selling mechanism where network operators assume the role of the sellers/bidders\footnote{Henceforth, sellers and bidders are used interchangeably unless stated otherwise.} and subscribers are the buyers of the transport services offered by the network operators. In this way, the Digital Marketplace strives to address the tussle between the actors involved, and thus, attempts to address both technological as well as economic constraints of the problem.

We create a simple economic model of the auction, and characterize the equilibrium under generic assumptions about the costs distributions of the network operators. Furthermore, the equilibrium is explicitly derived under more specific assumptions about the model; that is, two network operators and costs drawn from uniform distributions. In this case, we also analyse how the buyer can influence the bidding process in order to choose the network operator who matches their preferences about the service; for example, trading off quality for a lower price.

The rest of this chapter is organized as follows.

\section{Related Literature} % (fold)
\label{sec:related_literature_literature}
Over the last decade, several papers have explored the problem of intelligent network selection in heterogeneous wireless access networks. Antoniou et al., and Charlias et al.~model the problem as a noncooperative game between wireless access networks with the aim of obtaining the best possible trade-off between the efficiency and the available capacity of networks, while, at the same time, satisfying the requested quality by the subscribers \cite{Antoniou07, Charilas08}. Ormond et al.~propose an algorithm for cost-oriented and performance-aware network selection that maximizes consumer surplus \cite{OrmondCS106, OrmondCS206}. Niyato et al.~propose two algorithms based on evolutionary game theory for a network selection mechanism which performs intelligent load balancing so that network congestion and performance degradation can be avoided \cite{Niyato09}. Additionally, the same authors model the user churning behaviour in heterogeneous wireless access networks using evolutionary game theory \cite{NiyatoHossainConf2008}. Khan et al.~model the problem as a procurement second-price sealed-bid auction where network operators bid for the right to service the subscriber's request \cite{Khan110, Khan210}. Zhu et al.~build upon the work reported in~\cite{Niyato09}, and explore the dynamics of network selection, using Bayesian evolutionary game theory, in an environment where subscribers have only limited (incomplete) information about each others preferences \cite{ZhuNiyato2010}. Finally, Irvine et al.~propose a market-based framework called the Digital Marketplace where network operators compete in a variant of a procurement first-price sealed-bid auction for the right to transport the subscriber's requested service over their infrastructure \cite{DMLeBodic00, DMIrvine01, DMIrvine02}.
% section related_literature (end)

\section{The Digital Marketplace} % (fold)
\label{sec:the_digital_marketplace_literature}
The Digital Marketplace (DMP) is a market-based framework for trading wireless communications services. In its simplest form, there are four main groups of actors involved in the operation of the DMP: \emph{subscribers}, \emph{service providers}, \emph{network operators}, and \emph{market provider}. The subscribers are the end-users of the communications services. Furthermore, they can interact with the network operators either directly or through the service provider. In the former case, they act as the buyer in the DMP, and directly negotiate with the interested network operators. In the latter case, on the other hand, the service provider acts as an intermediary between the subscribers and the network operators, and hence, acts as the buyer during the negotiation process. Lastly, the market provider is tasked with operating the DMP; thus providing common platform for all actors involved. It is left open-ended who should be the market provider; however, one of the following three choices is the most likely: a regulatory body, a consortium of network operators, or a single network operator on behalf of the regulatory body \cite{DMIrvine02}.

The process of negotiation (or the network selection mechanism) in the DMP is based on a procurement first-price sealed-bid auction where the network operators represent the sellers/sellers, and the subscriber or the service provider is the buyer. Unlike in a standard procurement first-price sealed-bid auction, the winning bid is a weighted (convex) combination of both the network operator's monetary bid and their reputation rating; we will refer to it as the \emph{compound bid}. The network operator is elected as the winner of the auction if their compound bid is the lowest in value, and accrues their monetary bid minus the cost of transporting the service. The monetary bid is equivalent to the price of transporting the service by the network operator. The precise definition of the price is left open-ended; one possibility, for example, would be to charge the buyer per unit of bandwidth. The weights in the compound bid are set by the buyer before each auction, and are announced to the network operators. This effectively gives the buyer the freedom to choose any combination ranging from: a low price for the service but also poor quality; to a high quality but for a high price \cite{DMLeBodic00}.

Since the communications services are traded on an individual service level, it might be difficult for the buyer to judge the overall quality of the services supplied by a particular network operator \cite{DMIrvine02}. Therefore, the DMP maintains a reputation rating for each network operator. The reputation rating is directly proportional to the number of services that have prematurely been decommitted in the past by the respective network operator; i.e., the lower the number of the decommitted service requests, the lower the reputation rating \cite{DMLeBodic00}. The reputation of the respective network operator, on the other hand, is inversely proportional to the reputation rating; i.e., the lower the rating, the higher the reputation.

Although the DMP was developed when GSM was widespread and 3G was in its infancy, it can still be adapted to the current scenario where we witness the domination of packet-switched data communications over the traditional circuit-switched cellular telephony \cite{Ericsson2011}.
% section the_digital_marketplace (end)

% chapter literature (end)
